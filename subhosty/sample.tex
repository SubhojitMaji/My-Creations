\documentclass[a4paper]{book}
\usepackage{personal-mathphys}

\begin{document}
\maketitlepage{Documentation: personal-mathphys Package}{Package Reference Guide}{Mathematics and Physics Documentation}{December 2024}

\maketocpage

\chapter{Core Package Components}

\section{Document Setup and Fonts}
The package uses Palatino-like fonts (newpxtext) for body text and matching mathematics fonts (newpxmath). Headers use the Cabin sans-serif font.

\section{Page Layout}
Page geometry is optimized for mathematical content with:
\begin{itemize}
\item Margins: top/bottom 2.5cm, left/right 3cm
\item Header height: 14pt
\item Margin width: 3cm for annotations
\end{itemize}

\chapter{Mathematical and Physical Operators}

\section{Standard Math Operators}
\begin{definition}{Trace Operator}{trace}
The trace operator \(\Tr\) returns the sum of diagonal elements.
\end{definition}

\begin{definition}{Spectrum}{spec}
The spectrum operator \(\Spec\) gives eigenvalues of a matrix.
\end{definition}

\begin{definition}{Residue}{residue}
The residue operator \(\Res\) calculates complex function residues.
\end{definition}

\section{Vector Calculus Operators}
\begin{definition}{Curl}{curl-op}
The curl operator \(\curl\) measures rotation in vector fields.
\end{definition}

\begin{definition}{Divergence}{div-op}
The divergence operator \(\dive\) measures field expansion/contraction.
\end{definition}

\begin{definition}{Gradient}{grad-op}
The gradient operator \(\grad\) gives direction of steepest increase.
\end{definition}

\chapter{Custom Commands}

\section{Vector Notation}
\begin{definition}{Vector Commands}{vectors}
\begin{itemize}
\item \verb|\vect{v}|: Bold vector notation \(\vect{v}\)
\item \verb|\uvec{x}|: Unit vector notation \(\uvec{x}\)
\item \verb|\divg|: Divergence operator \(\divg\)
\end{itemize}
\end{definition}

\section{Quantum Mechanics}
\begin{definition}{Quantum Notation}{quantum}
\begin{itemize}
\item \verb|\bra{x}|: Bra notation \(\bra{x}\)
\item \verb|\ket{x}|: Ket notation \(\ket{x}\)
\item \verb|\braket{x}{y}|: Inner product \(\braket{x}{y}\)
\item \verb|\mean{A}|: Expectation value \(\mean{A}\)
\end{itemize}
\end{definition}

\section{Physics Constants}
\begin{definition}{Physical Constants}{constants}
\begin{itemize}
\item \verb|\clight|: Speed of light \(\clight\)
\item \verb|\planck|: Reduced Planck constant \(\planck\)
\item \verb|\boltz|: Boltzmann constant \(\boltz\)
\end{itemize}
\end{definition}

\chapter{Environments}

\section{Theorem-like Environments}
\begin{theorem}{Sample Theorem}{sample}
This demonstrates the theorem environment styling.
\end{theorem}

\begin{lemma}{Sample Lemma}{sample}
This demonstrates the lemma environment styling.
\end{lemma}

\begin{definition}{Sample Definition}{sample}
This demonstrates the definition environment styling.
\end{definition}

\section{Physics-Specific Environments}
\begin{law}{Newton's Second Law}{newton}
This demonstrates the physical law environment styling.
\end{law}

\begin{experiment}{Double-Slit}{slit}
This demonstrates the experiment environment styling.
\end{experiment}

\begin{note}{Important Note}{note}
This demonstrates the note environment styling.
\end{note}

\chapter{Color Schemes}

\section{Defined Colors}
The package defines these colors:
\begin{itemize}
\item NavyBlue (0,32,91)
\item DeepTeal (0,128,128)
\item WarmGray (128,128,128)
\item RoyalPurple (102,51,153)
\item TheoremBlue (230,240,255)
\item LemmaGreen (230,255,230)
\item DefinitionPeach (255,240,230)
\item ExampleLavender (245,230,255)
\item NotePink (255,230,240)
\end{itemize}

\end{document}
